
\chapter{Wiki generator tests}
\section{Paragraph test}
This is a simple test for the web/article generator.
It takes wiki pages and produces HTML and LaTeX output.
It also takes a content folder with html fragments and
embeds them on a scheletton.
The idea is to have the benefits of a wiki syntax
but maintain the files on a SVN server and editing
them using your favorite editor (vim?) instead of
the web interface.

Paragraph are separated by an empty line.
Like the one before.

\section{Lists tests}


\begin{enumerate}
	\item  first
	\item  second
\end{enumerate}

\begin{enumerate}
	\item  new first
\end{enumerate}

\begin{enumerate}
	\item  parent 1
	\begin{enumerate}
		\item  child 1
		\item  child 2
		\begin{enumerate}
			\item  subchild 1
		\end{enumerate}
		\item  child 3
	\end{enumerate}
	\item  parent 2
\end{enumerate}


Paragraph touching the itemization
\begin{itemize}
	\item  item
	\begin{enumerate}
		\item  inner numerated
		\begin{itemize}
			\begin{enumerate}
				\begin{itemize}
					\begin{enumerate}
						\item  Complex subitemization
					\end{enumerate}
				\end{itemize}
			\end{enumerate}
			\begin{itemize}
				\begin{itemize}
					\begin{enumerate}
						\item  Should go back to the common one
					\end{enumerate}
				\end{itemize}
			\end{itemize}
		\end{itemize}
	\end{enumerate}
\end{itemize}
Paragraph touching the itemization

\section{Special paragraphs}

\begin{quote}
Placing space or tabulators
creates a preformatted environment
	The first space is removed but the rest are kept
\end{quote}

\begin{abstract}
Some special paragraphs can be defined by preceding them by a
colon ended keyword such as 'Abstract:'
\end{abstract}

\begin{keywords}
And the Keywords: special paragraph
\end{keywords}

\section{Toc generation}

\subsection{Third level}

\subsubsection{Fourth level}

Headers generate labels.
They can be refered from a Table of Content.
This is the place holder for a Table of Content



The index header doesn't appear on the ToC.
The ToC in LaTeX output is ignored.

\section{Images and figures}
\label{OtherLabel}

Full featured figures can be generated by using the 'Figure:' special word
\begin{figure*}[htbp]
\begin{center}\includegraphics[width=4.8in]{MyImage.jpg}\end{center}
\caption{%
This is the caption text.
Til the next empty line.
}
\label{MyLabel}
\end{figure*}


\[
a\over{b}
\]

\begin{equation}
a\over{b}
\end{equation}

\marginpar{\footnotesize TODO: This is a pending task}


\section{Inline substitutions}

Normal {\em emphasis} Normal. {\em Another emphasis} Normal
Normal {\em em'phasis} Normal. {\em Another emphasis} Normal.
Normal {\bf bold} Normal. {\bf Another bold} Normal.
Normal {\bf bold {\em emphasis} more bold. don't } Normal. {\bf Another bold} Normal.
Normal {\em emphasis {\bf bold don't} more emphasis} Normal. {\bf Another bold} Normal.

a link\footnote{\hyperef{http://link.com}{http://link.com}}
\hyperef{http://link.com}{http://link.com} without alias

This work can be found in \cite{lee87}. And also in \cite{www-CLAM}.


$r_iO_i=r_jI_j$

Were $r_i$ and $r_j$ are the number of times that node i and node j will be fired during a period. The vector $\vec{r}$ gives the number of repetitions for each node.


